\documentclass[sigconf]{acmart}
%\documentclass[sigconf, authordraft]{acmart}

\usepackage{booktabs} % For formal tables
\usepackage{animate}
\usepackage{siunitx} % for SI formatting of digits and units

\usepackage[acronym,nonumberlist]{glossaries}
\makeglossaries

\input glossary/glossary.tex
\input glossary/acronyms.tex

% Copyright
%\setcopyright{none}
%\setcopyright{acmcopyright}
%\setcopyright{acmlicensed}
\setcopyright{rightsretained}
%\setcopyright{usgov}
%\setcopyright{usgovmixed}
%\setcopyright{cagov}
%\setcopyright{cagovmixed}


% DOI
%\acmDOI{xx.xxx/xxx_x}

% ISBN
%\acmISBN{xxx-xxxx-xx-xxx/xx/xx}

%Conference
\acmConference[PAW17]{PGAS Applications Workshop}{November 2017}{Denver, Colorado USA}
\acmYear{2017}
\copyrightyear{2017}

%\acmPrice{15.00}

\acmSubmissionID{xxx-xxx-xx}

\begin{document}

\copyrightyear{2017}
\acmYear{2017}
\setcopyright{acmlicensed}
\acmConference[PAW17]{PAW17: Second Annual PGAS Applications Workshop}{November 12--17, 2017}{Denver, CO, USA}
\acmPrice{15.00}
\acmDOI{10.1145/3144779.3169110}
\acmISBN{978-1-4503-5123-2/17/11}

\title[Performance portability of coarray-ICAR]{Performance portability of an intermediate-complexity atmospheric research model in coarray Fortran}
%\titlenote{Produces the permission block, and
  %copyright information}
\subtitle{Extended Abstract}
%\subtitlenote{The full version of the author's guide is available as
%  \texttt{acmart.pdf} document}

\author{
Damian Rouson$^1$, Ethan D Gutmann$^2$, Alessandro Fanfarillo$^2$, Brian Friesen$^3$\vspace{5pt}\\
\normalsize
{$^1$Sourcery Institute, USA}\vspace{1pt}\\
{$^2$National Center for Atmospheric Research, USA}\vspace{1pt}\\
{$^3$Lawrence Berkeley National Laboratory, USA}\vspace{1pt}\\
}
\renewcommand{\shortauthors}{Damian Rouson et al.}

%\author{Damian Rouson}
%\orcid{0000-0002-2234-868X}
%\affiliation{%
  %\institution{Sourcery Institute}
  %\streetaddress{2323 Broadway}
  %\city{Oakland}
  %\state{California}
  %\postcode{94612}
%}
%\email{damian@sourceryinstitute.org}
%\renewcommand{\shortauthors}{D. Rouson et al.}
%
%\author{Ethan D Gutmann}
%\orcid{0000-0003-4077-3430}
%\affiliation{%
  %\institution{National Center for Atmospheric Research}
  %\streetaddress{3450 Mitchell Ln.}
  %\city{Boulder}
  %\state{Colorado}
  %\postcode{80301}
%}
%\email{gutmann@ucar.edu}
%
%\author{Brian Friesen}
%\orcid{0000-0002-1572-1631}
%\affiliation{%
  %\institution{Lawrence Berkeley National Laboratory}
  %\streetaddress{1 Cyclotron Road, Mail Stop 59R4010A}
  %\city{Berkeley}
  %\state{California}
  %\postcode{94720}
%}
%\email{BFriesen@lbl.gov}

%\author{Alessandro Fanfarillo}
%\orcid{0000-0003-3487-7452}
%\affiliation{%
  %\institution{National Center for Atmospheric Research}
  %\streetaddress{ 1850 Table Mesa Dr.}
  %\city{Boulder}
  %\state{Colorado}
  %\postcode{80305}
%}
%\email{elfanfa@ucar.edu}

\begin{abstract}
We examine the scalability and performance of an open-source, \gls{caf} \gls{mini-app} that implements
the parallel, numerical algorithms that dominate the execution of \gls{icar}~\cite{gutmann2016intermediate} model
developed at the \gls{ncar}.
The Fortran 2008 \gls{mini-app} includes one Fortran 2008 implementation of a collective
subroutine defined in the Committee Draft of the upcoming Fortran 2018 standard.  The ability of \gls{caf} to run atop various
communication layers and the increasing \gls{caf} compiler availability facilitated evaluating several compilers,
runtime libraries and hardware platforms.  Results are presented for the GNU and Cray compilers, each of which offers
different parallel runtime libraries employing one or more communication layers, including \gls{mpi}, OpenSHMEM, and proprietary
alternatives.  We study performance on multi- and many-core processors in distributed memory.  The
results show promising scaling across a range of hardware, compiler, and runtime choices
on up to $\sim$100,000 cores.
\end{abstract}

%
% The code below was generated by the tool at
% http://dl.acm.org/ccs.cfm
%
\begin{CCSXML}
<ccs2012>
<concept>
<concept_id>10011007.10011006.10011008.10011009.10010175</concept_id>
<concept_desc>Software and its engineering~Parallel programming languages</concept_desc>
<concept_significance>500</concept_significance>
</concept>
<concept>
<concept_id>10010405.10010432.10010437.10010438</concept_id>
<concept_desc>Applied computing~Environmental sciences</concept_desc>
<concept_significance>300</concept_significance>
</concept>
</ccs2012>
\end{CCSXML}

\ccsdesc[500]{Software and its engineering~Parallel programming languages}
\ccsdesc[300]{Applied computing~Environmental sciences}

\keywords{coarray Fortran, computational hydrometeorology}

\maketitle

\input{body}

\bibliographystyle{ACM-Reference-Format}
\bibliography{bibliography}

\end{document}
